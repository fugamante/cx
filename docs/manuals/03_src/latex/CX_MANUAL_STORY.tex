\documentclass[11pt]{article}
% Shared preamble for cx manuals (Apple-clean, print-friendly).
% XeLaTeX recommended.
\usepackage[margin=1in]{geometry}
\usepackage{fontspec}
\usepackage{microtype}
\usepackage{hyperref}
\usepackage{xcolor}
\usepackage{enumitem}
\usepackage{titlesec}
\usepackage{fancyhdr}
\usepackage{listings}
\usepackage{setspace}

% Use macOS-bundled fonts for portability on this machine.
\setmainfont{Times New Roman}
\setsansfont{Helvetica}
\setmonofont{Menlo}

\definecolor{Ink}{RGB}{18,18,18}
\definecolor{Subtle}{RGB}{85,85,85}
\definecolor{Rule}{RGB}{220,220,220}
\definecolor{CodeBg}{RGB}{248,248,248}
\color{Ink}

\hypersetup{
  colorlinks=true,
  linkcolor=blue!45!black,
  urlcolor=blue!45!black
}

\lstdefinestyle{cx}{
  basicstyle=\ttfamily\small,
  backgroundcolor=\color{CodeBg},
  frame=single,
  framerule=0.2pt,
  rulecolor=\color{Rule},
  breaklines=true,
  columns=fullflexible,
  keepspaces=true,
  xleftmargin=0.6em,
  xrightmargin=0.6em
}

\setlist[itemize]{topsep=2pt,itemsep=1pt,parsep=0pt,leftmargin=1.2em}
\setlist[enumerate]{topsep=2pt,itemsep=1pt,parsep=0pt,leftmargin=1.6em}

\titleformat{\section}{\Large\bfseries}{\thesection}{0.6em}{}
\titleformat{\subsection}{\large\bfseries}{\thesubsection}{0.6em}{}
\titleformat{\subsubsection}{\normalsize\bfseries}{\thesubsubsection}{0.6em}{}

\pagestyle{fancy}
\fancyhf{}
\lhead{\textcolor{Subtle}{CX manual}}
\rhead{\textcolor{Subtle}{\thepage}}
\cfoot{}
\setlength{\headheight}{14pt}
\setlength{\parskip}{0.45em}
\setlength{\parindent}{0pt}


\title{cxcodex Manual (Story-First Walkthrough)}
\author{}
\date{Last updated: 2026-02-22}

\begin{document}
\maketitle
\vspace{-0.6em}

\section*{Premise}
You are in a repo, you want answers you can trust, and you want the runtime to stay predictable.
This walkthrough follows one realistic loop: \emph{capture context} \textrightarrow{} \emph{summarize}
\textrightarrow{} \emph{fan out} \textrightarrow{} \emph{run tasks} \textrightarrow{} \emph{optimize}.

\section{The Cast (What cx Actually Is)}
\begin{itemize}
  \item \textbf{bin/cx}: the one entrypoint you call. It routes Rust-first and falls back explicitly.
  \item \textbf{cxrs}: the canonical Rust engine. It owns capture, budgeting, schemas, quarantine, logging.
  \item \textbf{Bash}: compatibility/bootstrap only (\texttt{lib/cx/*.sh}).
  \item \textbf{.codex/}: the repo-local memory: schemas, logs, quarantine, tasks, state.
\end{itemize}

\section{Scene 1: You Enter the Repo}
You start by asking cx who it is \emph{today}: which backend is active, what the budgets are, and
where logs are going.

\begin{lstlisting}[style=cx]
cd ~/cxcodex
./bin/cx version
./bin/cx diag
\end{lstlisting}

What you should notice:
\begin{itemize}
  \item \texttt{execution\_path} is \texttt{rust:bin/cx} when cxrs is available.
  \item \texttt{log\_file} points at \texttt{.codex/cxlogs/runs.jsonl}.
  \item Budgets are explicit (chars/lines/clip\_mode).
\end{itemize}

\section{Scene 2: Capture Reality (Without Drowning the Model)}
The most common failure mode is context bloat: huge diffs, noisy logs, unbounded output.
cx treats system output as a first-class input with a strict pipeline:

\begin{lstlisting}[style=cx]
raw system output
  -> optional RTK routing (system output only)
  -> optional native reduction
  -> mandatory budgeting (clip/chunk)
  -> embed into prompt
\end{lstlisting}

Try a benign capture-driven call:
\begin{lstlisting}[style=cx]
./bin/cx cxo git status
\end{lstlisting}

If the repo is large and output is clipped, it should be \emph{visible} in telemetry:
\begin{lstlisting}[style=cx]
./bin/cx budget
./bin/cx trace
\end{lstlisting}

\section{Scene 3: Ask for a Deterministic Summary (Schema-Enforced)}
When you need a structured artifact (commit JSON, diff summary, next commands), you do \emph{not}
want prose. You want JSON-only, validated, and replayable.

\begin{lstlisting}[style=cx]
./bin/cx schema list
./bin/cx diffsum-staged | jq .
\end{lstlisting}

If the model returns invalid JSON, cx quarantines it and gives you a handle:
\begin{lstlisting}[style=cx]
./bin/cx quarantine list
./bin/cx quarantine show <id>
./bin/cx replay <id>
\end{lstlisting}

\section{Scene 4: Turn Work Into Tasks (Fan-out)}
When the objective is bigger than a single prompt, you fan out into small, role-tagged tasks.

\begin{lstlisting}[style=cx]
./bin/cx task fanout "Harden schema validation errors" --from staged-diff
./bin/cx task list --status pending
\end{lstlisting}

\section{Scene 5: Execute Tasks (Sequential for Now)}
You run a task (or run all pending tasks) through the same engine, and logs link the run to the task.

\begin{lstlisting}[style=cx]
./bin/cx task run task_001 --mode deterministic --backend codex
./bin/cx task run-all --status pending
\end{lstlisting}

\section{Scene 6: Stay Safe When Commands Are Suggested}
For semi-autonomy, suggestions are cheap. Execution is dangerous. The policy engine is the gate.

\begin{lstlisting}[style=cx]
./bin/cx policy show
./bin/cx policy check "rm -rf ."
\end{lstlisting}

\section{Scene 7: Learn From the Past (Optimize)}
The loop ends by asking: what is slow, what is token-heavy, and what is drifting?

\begin{lstlisting}[style=cx]
./bin/cx optimize 200
./bin/cx optimize 200 --json | jq .
\end{lstlisting}

\section{Appendix: Rules You Can Quote}
\begin{itemize}
  \item Budget system output. Always.
  \item Enforce schema for structured artifacts. Always.
  \item Quarantine failures; replay deterministically.
  \item Make telemetry a contract; validate it.
  \item Prefer small tasks over giant prompts.
\end{itemize}

\end{document}
