\documentclass[11pt]{article}
% Shared preamble for cx manuals (Apple-clean, print-friendly).
% XeLaTeX recommended.
\usepackage[margin=1in]{geometry}
\usepackage{fontspec}
\usepackage{microtype}
\usepackage{hyperref}
\usepackage{xcolor}
\usepackage{enumitem}
\usepackage{titlesec}
\usepackage{fancyhdr}
\usepackage{listings}
\usepackage{setspace}

% Use macOS-bundled fonts for portability on this machine.
\setmainfont{Times New Roman}
\setsansfont{Helvetica}
\setmonofont{Menlo}

\definecolor{Ink}{RGB}{18,18,18}
\definecolor{Subtle}{RGB}{85,85,85}
\definecolor{Rule}{RGB}{220,220,220}
\definecolor{CodeBg}{RGB}{248,248,248}
\color{Ink}

\hypersetup{
  colorlinks=true,
  linkcolor=blue!45!black,
  urlcolor=blue!45!black
}

\lstdefinestyle{cx}{
  basicstyle=\ttfamily\small,
  backgroundcolor=\color{CodeBg},
  frame=single,
  framerule=0.2pt,
  rulecolor=\color{Rule},
  breaklines=true,
  columns=fullflexible,
  keepspaces=true,
  xleftmargin=0.6em,
  xrightmargin=0.6em
}

\setlist[itemize]{topsep=2pt,itemsep=1pt,parsep=0pt,leftmargin=1.2em}
\setlist[enumerate]{topsep=2pt,itemsep=1pt,parsep=0pt,leftmargin=1.6em}

\titleformat{\section}{\Large\bfseries}{\thesection}{0.6em}{}
\titleformat{\subsection}{\large\bfseries}{\thesubsection}{0.6em}{}
\titleformat{\subsubsection}{\normalsize\bfseries}{\thesubsubsection}{0.6em}{}

\pagestyle{fancy}
\fancyhf{}
\lhead{\textcolor{Subtle}{CX manual}}
\rhead{\textcolor{Subtle}{\thepage}}
\cfoot{}
\setlength{\headheight}{14pt}
\setlength{\parskip}{0.45em}
\setlength{\parindent}{0pt}


\title{cxcodex Manual (Field Guide)}
\author{}
\date{Last updated: 2026-02-22}

\begin{document}
\maketitle
\vspace{-0.6em}

\section*{How to Read This}
This is a practical field guide: short rules, common failure modes, and ``do this next'' recipes.
It’s written for the moment you’re stuck, skeptical, or trying to keep costs under control.

\section{First Checks (30 Seconds)}
\begin{lstlisting}[style=cx]
./bin/cx version
./bin/cx diag
./bin/cx where version diffsum-staged task optimize
\end{lstlisting}

If \texttt{execution\_path} is not Rust, you are on fallback and should expect reduced guarantees.

\section{Capture and Context (The Budget Is the Product)}
\subsection*{Symptoms}
\begin{itemize}
  \item Runs become slow and expensive.
  \item Responses get vague (model is buried in noise).
  \item Cache hit rate degrades (prompts drift).
\end{itemize}

\subsection*{Fixes}
\begin{itemize}
  \item Reduce system output: prefer targeted \texttt{git diff -- <paths>} over whole-repo dumps.
  \item Enforce budgets: set \texttt{CX\_CONTEXT\_BUDGET\_CHARS} / \texttt{LINES} temporarily smaller to test clipping.
  \item Use chunking for fan-out: chunk output and turn it into parallelizable tasks.
\end{itemize}

\begin{lstlisting}[style=cx]
CX_CONTEXT_BUDGET_CHARS=2000 CX_CONTEXT_BUDGET_LINES=40 ./bin/cx cxo git status
./bin/cx chunk < some_big_output.txt
\end{lstlisting}

\section{Schemas (Determinism You Can Audit)}
\subsection*{What ``Schema-Enforced'' Means}
\begin{itemize}
  \item Prompt requests JSON-only output.
  \item Output parses as JSON.
  \item Output validates against \texttt{.codex/schemas/<name>.schema.json}.
  \item Failure triggers quarantine + non-zero exit.
\end{itemize}

\begin{lstlisting}[style=cx]
./bin/cx schema list
./bin/cx commitjson | jq .
./bin/cx diffsum-staged | jq .
\end{lstlisting}

\subsection*{If It Fails}
\begin{lstlisting}[style=cx]
./bin/cx quarantine list
./bin/cx replay <quarantine_id>
\end{lstlisting}

\section{Safety (You Are Not Your Shell History)}
\subsection*{Block List (Minimum)}
\begin{itemize}
  \item \texttt{sudo}, \texttt{rm -rf}, \texttt{curl | bash}
  \item chmod/chown on system paths
  \item writes outside repo root
\end{itemize}

\begin{lstlisting}[style=cx]
./bin/cx policy show
./bin/cx policy check "sudo rm -rf /"
\end{lstlisting}

\section{Tasks (Parallel-Ready, Single-Worker Today)}
Use tasks when the objective is bigger than one bounded prompt.

\begin{lstlisting}[style=cx]
./bin/cx task add "Tighten schema errors" --role implementer
./bin/cx task fanout "Ship release note improvements" --from staged-diff
./bin/cx task run-all --status pending
\end{lstlisting}

\section{Telemetry and Logs (Trust, but Verify)}
\begin{lstlisting}[style=cx]
./bin/cx logs validate --fix=false
./bin/cx metrics 50
./bin/cx trace
\end{lstlisting}

\subsection*{Interpretation Cheatsheet}
\begin{itemize}
  \item \texttt{effective\_input\_tokens} high: prompts too large or too variable.
  \item \texttt{clipped=true} often: reduce capture scope or raise budgets intentionally.
  \item cache trend down: stabilize prompt templates and system capture.
\end{itemize}

\section{Optimize (Make It Cheaper, Faster, Safer)}
\begin{lstlisting}[style=cx]
./bin/cx optimize 200
./bin/cx optimize 200 --json | jq .
\end{lstlisting}

\end{document}
